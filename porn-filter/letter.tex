\documentclass{letter}
\usepackage[a4paper]{geometry}
\signature{Taylor Hornby}
\address{Taylor Hornby \\ XXXXXXXXXXXXXXXXX \\ XXXXXXX \ XXXXXXX}
\begin{document}
\begin{letter}{Michelle Rempel \\ House of Commons\\ Ottawa, ON \ K1A 0A6}
\opening{Dear Michelle Rempel:}

Your colleague Joy Smith is proposing a mandatory opt-out Internet porn filter.
As a computer scientist and Internet citizen, it is my duty to point out why
this is a horrible idea.

I'm not a fan of slippery-slope arguments, but it applies here. Is it just porn,
or will the government expand what's being filtered once the infrastructure
exists? The UK's filter was supposed to filter porn, but includes by default
things like alcohol/smoking and web forums (harmless online discussion
communities). I am also concerned that once the filtering infrastructure exists,
it can be abused by the copyright industry to censor competing artists and
so-called pirates that make fair use of their content.

Defining pornography is difficult. ``I know it when I see it'' is not a computer
algorithm. If the definition is too lose, there will be many false positives.
Educational images, rape survivor support groups, and non-pornographic criticism
of the political parties supporting the porn filter will be censored. The
definition will almost certainly not be decided upon in a democratic way.
Someone will be appointed to create the formal definition, and the people will
have no say.

Even if we can agree on an English-language definition of porn, implementing it
into an automated filtering system is technically impossible. Even the state of
the art in artificial intelligence cannot produce an algorithm that decides
whether some web page meets the definition of pornography or not. There will
always be \emph{lots} of false positives and false negatives. Even if the system
is correct 99\% of the time, out of a hundred million web pages, a million
innocent and benign pages will be blocked. Actual systems cannot attain anything
like 99\% accuracy, so in practice, it will be much worse.

Mrs. Smith repeatedly dismisses criticisms with the argument ``if you don't want
it on, you can turn it off.'' There are several counterexamples to this claim:

\begin{itemize}
    \item Adults living with other adults. One may be too embarassed to ask the
    other to disable the filter, and as a result, lives with the censorship.
    \item Employers, public libraries, Internet cafes, WiFi hotspots are
    unlikely to disable the filter to get around one false positive.
    \item The ISP's settings interface may not be accessible, so users with
    disabilities may not be able to disable the filter.
\end{itemize}

I do not believe that it is Mrs. Smith's intention to censor the Internet, but
all filtering systems have false positives, and the false positives constitute
censorship.

Porn filtering harms people in other countries, who have no right to vote on
this issue. How is it fair to block access to an American company's website when
they have no democratic ability to oppose the filtering? The Internet is global.
It is independent of any state. Filtering it here damages it for everyone.

The filter will also be used maliciously as a censorship tool. Posting porn on
a competitor's website will get them blocked, and it's not clear what recourse
they will have to get themselves unblocked. Big business will use this technique
to take down smaller businesses that don't have the legal resources to fight the
censorship.

There are already many tools parents can use to filter their children's Internet
access. Examples are K9 Web Protection, X3 Watch, NetGenie, iBoss, BSsafe,
IamBigBrother, etc. From an Internet engineering point of view, the ISP is the
wrong place to implement the filter. The filter should be implemented as near to
the computer as possible (installed on the computer itself or on the local
network), to minimize negative effects. This is already easy for parents to do,
using the previously-mentioned tools.

The Canadian Charter of Rights and Freedoms guarantees my right to freedom of
``expression, including freedom of the press and other media of communication.''
An opt-out porn filter would (intentionally or not) violate these rights.

These are only some of the many reasons Mrs. Smith's proposal is a bad idea.
I would like to know your thoughts on this issue. If you could please make them
public or reply to this letter, I would really appreciate it.


\closing{Thanks,}
\end{letter}
\end{document}
